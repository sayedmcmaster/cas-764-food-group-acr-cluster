\section{Related Work}

\flushleft \justifying In this section, I will provide a literature survey on the association of Food Groups with ACR concerning CKD where data clustering is also utilized. The majority of the research focused on the effect of Nutrients and Drugs on CKD progression. Food group and food subgroup-based study as well as a whole eating pattern-based study is a recent trend. In terms of related work, different categories of related work can be seen to be relevant such as the effects of drugs and nutrients on CKD progression, drugs and nutrients on ACR and GFR, Food Groups/subgroups on CKD progression, Food Groups/subgroups on ACR and GFR, as well as where the above was done by data clustering as well. However, I will primarily focus on studies of Food Groups on ACR. Additionally, I will focus on the usage of clustering in CKD-related studies irrespective if that is for Food Groups or ACR related or not.

\subsection{Food Groups relation with CKD and ACR}
\flushleft \justifying A study ~\cite{Chen2016} studied the relation of plant protein for all-cause CKD mortality. It used statistical models and regression (Cox). It found lower mortality with a high intake of plant protein. Another study ~\cite{Liu2019} used multivariable logistic regression analysis to study vegetarian diets and CKD relations. It found vegetarian diet can be protective. However, one of my past projects as well as other studies ~\cite{Aleix2019} showed that vegetables can also be a risk factor because of high potassium ( and pesticide) content. One research ~\cite{Gutierrez2014} studied five eating patterns and found that processed and fried food are harmful to CKD patients whereas eating patterns with fruits and vegetables are protective. The five eating patterns are convenience (Chinese and Mexican foods, pizza), Plant-based, Sweets/Fats, Southern, and Alcohol/Salads. A study ~\cite{Huang2013} found that the Mediterranean diet has a lower likelihood of having CKD in elderly men. It ~\cite{Huang2013} used unpaired t-test, nonparametric Mann–Whitney test, or chi-square test to Compare CKD and non-CKD men. ~\cite{Ricardo2013} used Cox proportional hazards models with adjustment for covariates and found that healthy lifestyles have a lower risk of all-cause mortality in CKD. A study by Suruya et. al. ~\cite{Tsuruya2015} found that an unbalanced diet is more likely to create adverse clinical outcomes for CKD patients. It used a principal components analysis (PCA) with Promax rotation to derive a smaller set of food groups for analysis. It also used Cox regression with various combinations of covariants. ~\cite{Ricardo2015} found that adherence to a healthy lifestyle decreases risk. ~\cite{Asghari} used multivariable logistic regression and found that high fat and high sugar increase the incidence of CKD ~\cite{ahmed2018}.

\subsection{CKD Studies with Data Clustering}
\flushleft \justifying A study on CKD such as ~\cite{Zheng2021} has utilized an unsupervised consensus clustering on 72 baseline characteristics. Depending on patient characteristics clusters were created and clusters/groups of different risks of CKD progression are identified. However, this is not a study on Food Group association with CKD/ACR. ~\cite{LockWood2020} used 11 items on the Kidney Disease Quality of Life symptom profile to identify patient subgroups. This was based on similar observed physical symptom response patterns. However, the goal was to identify subgroups with differing severity. However, this is not a study on Food Group association with CKD/ACR.

\flushleft \justifying Now, I did not see a study utilizing clustering to see how Food Groups are associated with CKD/ACR progression. I also did not come across studies utilizing clustering to find vulnerable groups in terms of sensitivity to food groups for ACR readings. However, a more comprehensive literature survey will be required to conclude this.