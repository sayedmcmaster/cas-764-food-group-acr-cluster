\section{Conclusion}

\flushleft \justifying %Chronic Kidney Disease (CKD) is highly prevalent today. CKD can lead to kidney failure and death. CKD has no permanent cure. Hence, the treatment options involve drugs, lifestyle, and food choices. The majority of the research studied the effects of Drugs and nutrients on CKD progression. The effect of food and eating patterns on CKD has attracted recent attention. However, the research is not significant yet. Hence,
\flushleft \justifying In this study, I have studied the effect of food groups on a CKD diagnostic marker called Urine Albumin Creatinine Ratio (ACR). Urine ACR can determine the severity of CKD. This study finds out how Food Groups affect ACR readings and also the severity in CKD patients. Moreover, we also identified age groups and CKD stages that are the most sensitive to food groups. We have utilized Age and ACR reading-based Clusters and also explored on K-Means clustering. In this study, we have utilized Demographics, Dietary Intake, and ACR datasets from NHANES, and CDC. In this study, I have utilized Principle Component Analysis (PCA) to justify the data and find the important features such as Food Groups in the data. Afterward, I utilized Pearson correlation and regression to study the relation of Food Groups with ACR. My experiments show no significant relationship between Food Groups and ACR in the total population. I also found that Food Groups show more correlation with ACR for groups with ages under 30 and ACR readings between 3 to 30. I also have applied several regression techniques such as Linear and Polynomial Regression, Linear and Polynomial Regression with Cross Validations, and Bayesian and Random Forest Regression with or without Cross Validations to find out the regression coefficients of the food groups affecting ACR. Experiments show that Higher Energy (Calorie) intake can lead to increased ACR readings. All Regression techniques show the same for calories. Afterward, all regression techniques showed that Poly Unsaturated Fatty Acids have the 2nd highest contribution to ACR for Cluster 2. All regression techniques also showed that Cholesterol has the 3rd highest contribution to ACR for Cluster 2.

\subsection{Future Work}
\flushleft \justifying Several immediate future extensions of this study can include integrating data from 1996, utilizing USDA food groups, conducting experiments on the KMeans clustered data that I generated, or utilize Consensus Clustering to cluster data, and using Machine Learning to predict ACR from dietary intake.

\flushleft \justifying The research focus can also include finding groups that are the most sensitive to food groups for ACR and GFR. Studies can place the combined effect on ACR/GFR as the target variable to find relations with food groups and sub-groups. I did not come across studies that study food groups' effect on the combined GFR and ACR metrics. Additional studies can focus on how food groups' effect on ACR/GFR propagates to Mortality and Survival of CKD patients. Survival and Mortality can use food groups and sub-groups as the input variables or can use the effect on ACR/GFR metrics as the input.