\section{Abstract}

\flushleft \justifying Chronic Kidney Disease (CKD) is very prevalent today. Thirty-seven million Americans currently have Chronic Kidney Disease (CKD). CKD can lead to kidney failure and death. In addition, CKD is also a primary cause of death from stroke and heart disease. CKD has no permanent cure. Hence, the treatment options involve drugs, lifestyle, and food choices. The majority of the research studied the effects of Drugs and nutrients on CKD progression. The effect of food and eating patterns on CKD has attracted recent attraction. However, the research is not significant yet. Additional research is required to understand how Food and Lifestyle choices affect CKD. Hence, in this study, I have studied the effect of food groups on a CKD diagnostic marker called Urine Albumin Creatinine Ratio (ACR). This study finds out how Food Groups affect ACR readings in CKD patients. Moreover, we also wanted to identify what clusters of the population such as clusters based on age groups and CKD stages are the most sensitive to food groups. Hence, we studied the effect of Food Groups on ACR for different age groups and ACR-induced CKD stages. Our experiments show no significant relationship between ACR and Food Groups in the population as a whole. We found that Food Groups show more correlation with ACR for the cluster with ages under 30 and ACR readings between 3 to 30 than the other clusters. In this study, I have utilized Demographics, Dietary Intake, and ACR datasets from NHANES, and CDC. In my study, I have utilized Principle Component Analysis (PCA) to find important features in the data. Afterward, I utilized Pearson correlation and regression to study the relation of Food Groups with ACR for the groups. I also have applied several regression techniques such as LinearRegression, Polynomial Regression, Bayesian Regression, and Random Forest Regressor to find out the regression coefficients of the food groups affecting ACR. Our studies show that Higher Energy (Calorie) intake can lead to increased ACR readings. All Forms of Regression show the same. Afterward, all regression types show that Poly Unsaturated Fatty Acids have the 2nd highest contribution to ACR for Cluster 2. Then, all regression types show that Cholesterol has the 3rd highest contribution to ACR for Cluster 2. \footnote{GitHub Repository of code and data: https://github.com/sayedmcmaster/cas-764-food-group-acr-cluster . However, data files over 25 MBs are either uploaded as a zip file or did not get uploaded. Code files will need to be placed in the proper folder for them to execute. I may have put code files at the root for easy viewing.}

%Urine ACR determines the severity of CKD in stages such as Stage 1 (0, $<$ 3), Stage 2 (3, 30), and Stage 3 (30+). 